\documentclass[journal,final]{../inc/oldtran/IEEEtran}
\usepackage{amsmath,amssymb,graphicx,subfig}
\usepackage[usenames,dvipsnames]{color}
\usepackage{algorithm,algorithmic}
\newcommand{\E}{\textrm{E}}

\begin{document}

\title{Inference in Hidden Markov Models with Explicit State Duration Distributions}
\author{Michael Dewar, Chris Wiggins, Frank Wood\thanks{This work was supported in part by a grant from the Xerox research foundation.}}%
%

    
\maketitle

\begin{abstract}
Explicit-state-duration hidden Markov models (EDHMM) are HMMs that have latent states consisting of both discrete state-indicator and discrete state-duration random variables.  In contrast to the implicit geometric state duration distribution possessed by the standard HMM, EDHMMs allow the direct parameterisation and estimation of per-state duration distributions. As most duration distributions are defined over the positive integers, truncation or other approximations are usually required to perform EDHMM inference.  In this letter we borrow from the inference techniques developed for unbounded state-cardinality (nonparametric) variants of the HMM and use them to develop a tuning-parameter free, black-box inference procedure for EDHMMs.  %We illustrate the performance gains of this approach using synthetic data.
\end{abstract}

\section{Introduction}

\label{section}
Hidden Markov models (HMMs) are a fundamental tool for data analysis and exploration.  Many variants of the basic HMM have been developed in response to shortcomings in the original HMM formulation \cite{Rabiner89}.  In this paper we address inference in the explicit state duration HMM.  By state duration we mean the amount of time an HMM dwells in a state.  In the standard HMM specification, a state's duration is implicit and, a priori, distributed geometrically.

The explicit state duration HMM (EDHMM) \cite{Rabiner89} was developed to allow explicit parameterization and direct inference of state duration distributions.  Approximate EDHMM estimation and inference can, in some cases, be performed using a modified forward-backward algorithm; specifically if either the sequence is short or a tight ``allowable'' duration interval for each state is hard-coded a priori \cite{Yu2006}.   If the sequence is short then forward-backward can be run on a state representation that allows for all possible durations shorter than the observed sequence length.  If the sequence is long then forward-backward only remains computationally tractable if only transitions between ``allowed'' durations (those that lie within pre-specified allowable intervals) are considered.   If the true state durations lie outside those intervals then the resulting model estimates will be incorrect in the sense that the learned duration distributions will reflect only what is allowed given the pre-specified duration intervals (which may poorly describe the true duration distributions).

Our contribution is the development of a procedure for EDHMM inference that does not require any hard pre-specification of duration intervals and, as a result, is no longer approximate.  The technique we use to do this 
is borrowed from sampling procedures developed for nonparametric  Bayesian HMM variants \cite{vanGael2008}.  Our key insight is simple: the machinery developed for doing inference in HMMs with a countable number of states is precisely the same as that which is needed for doing inference in an EDHMM with duration distributions over countable support.  So, while the EDHMM is a distinctly parametric model, the tools from nonparametric Bayesian inference can be applied such that black-box inference becomes possible and, in fact, efficient.

In this work we show specifically that a ``beam-sampling'' approach  \cite{vanGael2008} works for non-approximate estimation of EDHMMs, learning both the transition structure and duration distributions simultaneously.

In demonstrating our EDHMM inference technique we consider a synthetic system  in which the state-cardinality is known and finite, but where each state's duration distribution is unknown. We show that the EDHMM beam sampler performs accurate tracking whilst capturing the duration distributions as well as the probability of transitioning between states.

The remainder of the letter is organised as follows.  In Section~\ref{sec:Model} we introduce the EDHMM; in Section~\ref{sec:inference} we review beam-sampling for the infinite Hidden Markov Model (iHMM) \cite{Beal2002} and show how it relates to the EDHMM inference problem; and in Section~\ref{sec:experiments} we show results from using the EDHMM to model synthetic data.  


    \begin{figure}[t]
        \centering
        \subfloat[][]{
            \includegraphics[width=0.23\textwidth]{../pic/EDHMM_graphical_model.pdf}
            \label{fig:graphical model}
        }
        \subfloat[][]{
            \includegraphics[width=0.23\textwidth]{../pic/EDHMM_aux_graphical_model.pdf}
         \label{fig:aux graphical model}
        }
        \caption{a) The Explicit Duration Hidden Markov Model. The time left in the current state $x_t$ is denoted $d_t$. The observation at each point in time is denoted $y_t$. b) The EDHMM with the additional auxiliary variable $u_t$ used in the beam sampler.}
    \label{fig:graphs}    
    \end{figure}
\section{Explicit Duration Hidden Markov Model}
\label{sec:Model}

The EDHMM captures the relationships among state $x_t$, duration $d_t$, and  observation $y_t$ over time $t$. It consists of four components: the initial state distribution, the transition distributions, the observation distributions, and the duration distributions. 

We define the observation sequence $\mathcal{Y} = \{y_1, y_2, \ldots, y_T\}$; the latent state sequence $\mathcal{X} = \{x_0, x_1, x_2, \ldots, x_T\}$; and the remaining time in each segment $\mathcal{D} = \{ d_1, d_2, \ldots, d_T\}$, where $x_t \in \{ 1, 2, \ldots, K\}$ with $K$ the maximum number of states, $d_t \in \{1, 2, \ldots \}$, and $y_t \in \mathbb{R}^n$.     We assume that the Markov chain on the latent states is homogenous, i.e., that $p(x_t = i | x_{t-1}=j, A) = a_{i,j} \forall t$ where $A$ is a $K\times K$ matrix with element $a_{i,j}$ at row $i$ and column $j.$  The prior on $A$ is row-wise Dirichlet with zero prior mass on self-transitions, i.e.  $p(a_{i,:}) = \mathrm{Dir}({1}/{K-1}, \ldots, 0 , \ldots {1}/{K-1})$ where $a_{i,:}$ is a row vector and the $i$th Dirichlet parameter is $0.$  Each state is imbued with its own duration distribution $p(d_t | x_t = k) = p(d_t | \lambda_{k})$ with parameter $\lambda_k$.  Each duration distribution parameter is drawn from a prior $p(\lambda_k)$ which can be chosen in an application specific way.  The collection of all duration distribution parameters is $\lambda = \{\lambda_1, \ldots, \lambda_K\}$.  Each state is also imbued with an observation generating distribution $p(y_t  | x_t = k) = p(y_t | \theta_{k})$ with parameter $\theta_k$.  Each observation distribution parameter is 
%given 
drawn from a prior $p(\theta_k)$ also to be chosen according to the application.   The set of all observation distribution parameters is $\theta.$  In the following exposition, explicit conditional dependencies on component distribution parameters are omitted to focus on the particulars unique to the EDHMM.

In an EDHMM the transitions between states are only allowed at the end of a segment:
    \begin{equation}
        p(x_t | x_{t-1}, d_{t-1}) = 
        \begin{cases} 
            \delta(x_t, x_{t-1}) & \textrm{if $d_{t-1} > 1$} \\
            p(x_t | x_{t-1}) & \textrm{otherwise}
        \end{cases}
    \end{equation}
where the Kronecker delta $\delta(a,b) = 1$ if $a=b$ and zero otherwise. The duration distribution generates segment lengths at every state switch:
    \begin{equation}
        p(d_t | x_{t}, d_{t-1}) = 
        \begin{cases} 
            \delta(d_t, d_{t-1}-1) & \textrm{if $d_{t-1} > 1$} \\
            p(d_t | x_{t}) & \textrm{otherwise.}
        \end{cases}
    \end{equation}
The joint distribution of the EDHMM is 
\begin{multline}
    \label{eq:joint}
    p(\mathcal{X},\mathcal{D},\mathcal{Y}) =  \\ 
    p(x_0)p(d_0)\prod_{t=1}^T p(y_t | x_t, \theta) p(x_t | x_{t-1}, d_{t-1}, A) p(d_t | x_{t}, d_{t-1}, \lambda)
\end{multline}
corresponding to the graphical model in Figure \ref{fig:graphical model}. Alternative choices to define the duration variable $d_t$ exist; see \cite{Chiappa2011} for details. Algorithm \ref{alg:gen} illustrates the EDHMM as a generative model.

\begin{figure*}[ttt!]
\begin{minipage}[t]{2in}
\begin{algorithm}[H]
    \caption{Generate Data}
    \label{alg:gen}
    \begin{algorithmic}
        \STATE sample $x_0 \sim p(x_0)$, $d_0 \sim p(d_0)$
        \FOR {$t = 1, 2, \ldots, T$}
            \IF{$d_{t-1} = 1$}
                \STATE a new segment starts:
                \STATE sample $x_t \sim p(x_t | x_{t-1})$
                \STATE sample $d_t \sim p(d_t | x_t)$
            \ELSE
                \STATE the segment continues:
                \STATE $x_t = x_{t-1}$
                \STATE $d_t = d_{t-1} - 1$
            \ENDIF
        \STATE sample $y_t \sim p(y_t | x_t)$
        \ENDFOR
    \end{algorithmic}
\end{algorithm} 
 \end{minipage}
 \hfill
 \begin{minipage}[t]{3.3in}
\begin{algorithm}[H]
    \caption{Sample the EDHMM}
    \label{alg:beam}
    \begin{algorithmic}
%\STATE \textbf{Initialise:}
\STATE Initialise parameters $A$, $\lambda$, $\theta.$ Initialize $u_t$ small $\forall\, T$
%\STATE set $\mathcal{U}$ to a small value for all $u_t \in \mathcal{U}$
\FOR{sweep $ \in \{1,2,3,\ldots \}$}
   % \STATE \textbf{Forwards pass}:
    \STATE \textbf{Forward}: run \eqref{eqn:scaled forward} to get $\hat{\alpha}_t(z_t)$ given $\mathcal{U}$ and $\mathcal{Y} \; \forall\, T$
    %\STATE \textbf{Backwards sample}:
    \STATE \textbf{Backward}: sample $z_T \sim \hat{\alpha}_T(z_T)$
    \FOR{$t \in \{T, T-1, \ldots, 1\}$}
        \STATE sample $z_{t-1} \sim \mathbb{I}(u_t < p(z_{t} | z_{t-1}))\hat{\alpha}_{t-1}(z_{t-1})$
    \ENDFOR
\STATE \textbf{Slice:}\FOR {$t \in \{1, 2, \ldots, T \}$}
    \STATE evaluate $l = p(d_t|x_t,d_{t-1})p(x_t|x_{t-1},d_{t-1})$
    \STATE sample $u_{t} \sim \mathrm{Uniform}(0,l)$
    \ENDFOR
\STATE sample parameters $A$, $\lambda$, $\theta$
\ENDFOR
\end{algorithmic}
\end{algorithm}
\end{minipage}
 \hfill
\end{figure*}

\section{EDHMM Inference}

\label{sec:inference}

Our aim is to estimate the conditional posterior distribution of the latent states ($\mathcal{X}$ and $\mathcal{D}$) and parameters ($\theta, \lambda$ and $A$) given observations $\mathcal{Y}$ by samples drawn via Markov chain Monte Carlo. Sampling $\theta$ and $A$ given $\mathcal{X}$ proceeds per usual textbook approaches \cite{Bishop06}.  Sampling $\lambda$ given $\mathcal{D}$ is straightforward in most situations.  Gibbs sampling  $\mathcal{X}$ is possible but, for reasons similar to those in the case of graphical models with chain dependency structure, one would not expect Gibbs sampling to work well for this model.  The main contribution of this paper is to show how to generate posterior samples of  $\mathcal{X}$ and $\mathcal{D}$ efficiently given all other random variables.

\subsection{Forward Filtering, Backward Sampling}

We can, in theory, use the forward messages from the forward backward algorithm \cite{Rabiner89} to sample the conditional posterior distribution of $\mathcal{X}$ and $\mathcal{D}.$   To do this we treat each state-duration tuple as a single random variable 
%(i.e.~$z_t = \{x_t,d_t\}$).  
(introducing the notation $z_t = \{x_t,d_t\}$).  
Doing so recovers the standard hidden Markov model structure and hence standard forward messages can be used directly.  A forward filtering, backward sampler for $\mathcal{Z} = \{z_1, \ldots, z_T\}$ conditioned on 
%everything else 
all other random variables
requires the classical forward messages:
    \begin{equation}
        \alpha_t(z_t) = 
        \sum_{z_{t-1}}
        p(z_t | z_{t-1}) 
        p(y_t|z_t) 
        \alpha_t(z_{t-1}) 
        \label{eqn:forward recursion}
    \end{equation}
   % \begin{equation}
   % \beta(z_t) = \sum_{z_{t+1}} p(z_{t+1} | z_t) p(y_{t+1}|z_{t+1}) \beta_{t+1}(z_{t+1})
   % \label{eqn:backward recursion}
    % \end{equation}
     where the transition probability can be factorised according to our modelling assumptions:
     \begin{equation}
        p(z_{t} | z_{t-1}) = p(x_t | x_{t-1}, d_{t-1}) p(d_t | d_{t-1}, x_t).
     \end{equation}

Unfortunately the sum in \eqref{eqn:forward recursion} has an infinite number of terms in the usual case of  duration distributions with countably infinite support. The standard approach to EDHMM inference involves truncating these sums in order to make them tractable. 
%If all durations are truncated at a maximum $d_\mathrm{max}$, yet many states are of duration far less than $d_{\mathrm{max}}$, this necessarily leads to much unnecessary computation
% if not all states have durations near $d_\mathrm{max}$. 
A reasonable computational artifice is to truncate all durations at some maximum $d_\mathrm{max}$, or to consider all possible durations up to the observed length of the sequence. This leads to per sample, forward-backward computational complexity of $O(T(Kd_\mathrm{max})^2)$  and  $O(T(KT)^2)$ respectively.  Alternatively one could risk inference that will simply fail if an actual duration lies outside hard-coded allowable duration intervals $d_\mathrm{min}<d<d_\mathrm{max}$.  This yields $O(T(K(d_\mathrm{max}-d_\mathrm{min}))^2)$ complexity.
The beam-sampler we propose behaves like a dynamic version of the latter, automatically scaling the allowable per-state duration intervals.   Its use of slice-sampling auxiliary variables results in an asymptotically exact sample with no risk of incorrect inference resulting from incorrectly pre-specified duration bounds.   We do not characterize the computational complexity of the proposed beam sampler in this work but note that it is upper bounded by $O(T(KT)^2)$ (i.e., the beam sampler admits durations of length equal to the entire sequence) but in practice is found to be as  or more efficient than the doubly hard-coded, risky approach.

%would be achieved with a uniform duration distribution with support over $[1, d_\mathrm{max}]$. 

\subsection{EDHMM Beam Sampling}

A recent contribution to inference 
in 
%is 
the infinite Hidden Markov Model (iHMM) \cite{Beal2002} suggests a way around truncation \cite{vanGael2008}.  The iHMM is an HMM with a countable number of states.  Computing the forward message for a forward filtering, backward sampler for the latent states in an iHMM also requires a sum over a countable number of elements.  
% which can be dealt with using the beam sampler \cite{vanGael2008}. Here, instead of a %potentially infinite duration-space cardinality, the iHMM faces a potentially infinite state cardinality. 
%The beam sampler combines dynamic programming with slice sampling \cite{Neal2003} in order to sample from the hierarchical Dirichlet process inherent to the iHMM.
The ``beam sampling'' approach  \cite{vanGael2008}, which we can apply largely without modification, is to truncate this sum by introducing a ``slice'' \cite{Neal2003} auxiliary variable $\mathcal{U} = \{u_1, u_2, \ldots,u_T\}$ at each time step.  The auxiliary variables are chosen in such a way as to automatically limit each sum in the forward pass to a finite number of terms while still allowing all possible durations.% in the asymptotic sampling limit.  

%which is used in a Gibbs sampling scheme to allow sampling from the true posterior state transition distribution.
%In the context of the EDHMM, the beam sampler allows us to sample directly from the invariant posterior $p(\mathcal{X},\mathcal{D},\mathcal{U} | \mathcal{Y})$ using standard Gibbs sampling theory. Using a block-Gibbs sampling scheme, $\mathcal{U}$ is sampled so that we are able to sample from $p(\mathcal{X},\mathcal{D} | \mathcal{U}, \mathcal{Y})$ and $p(\mathcal{U} | \mathcal{X},\mathcal{D})$. Upon convergence of the Markov chain, by simply marginalising $\mathcal{U}$ the remaining samples of state and duration are drawn directly from the posterior, without any need to choose a maximum duration. 

The particular choice of auxiliary variable $u_t$ is important.  We follow  \cite{vanGael2008} in choosing $u_t$ to be conditionally distributed given the current and previous state and duration in the following way (see the graphical model in Figure \ref{fig:aux graphical model}):
\begin{equation}
    \label{eqn:slice}
    p(u_t | z_t, z_{t-1}) = 
    \frac
    {\mathbb{I}(0 < u_t < p(z_t | z_{t-1}))} 
    {p(z_t | z_{t-1})}.
\end{equation}
where $\mathbb{I}(\cdot)$ returns one if its operand is true and zero otherwise. Given $\mathcal{U}$ it is possible to sample the state $\mathcal{X}$ and duration $\mathcal{D}$ conditional posterior. 
%\begin{equation}
%    \hat{\alpha}_t(z_t) = p(z_t, d_t , \mathcal{Y}_1^t, \mathcal{U}_1^t)
%\end{equation}
Using notation $\mathcal{Y}_{t_1}^{t_2} = \{y_{t_1}, y_{t_1+1}, \ldots,y_{t_2}\}$  to indicate sub-ranges of a sequence, the new forward messages we compute are:
\begin{eqnarray}
   \hat{\alpha}_t(z_t) &=& 
   p(z_t, \mathcal{Y}_1^t , \mathcal{U}_1^{t})   \label{eqn:scaled forward} \\
   &=& 
   \sum_{z_{t-1}}
   p(z_t, z_{t-1} , \mathcal{Y}_1^t , \mathcal{U}_1^{t}) \nonumber \\
   &\propto& 
   \sum_{z_{t-1}}
   p(u_{t} | z_t, z_{t-1})
   p(z_t, z_{t-1} , \mathcal{Y}_1^t, \mathcal{U}_1^{t-1}) \nonumber \\
   &=& 
   \sum_{z_{t-1}}
   \mathbb{I}(0 < u_{t} < p(z_t | z_{t-1}))
   p(y_t|z_t) \hat{\alpha}_{t-1}(z_{t-1}) \nonumber.
\end{eqnarray}
The indicator function $\mathbb{I}$ results in non-zero probabilities in the forward message for only those states $z_t$ whose likelihood given $z_{t-1}$ is greater than $u_t$. The beam sampler derives its computational advantage from the fact that the set of $z_t$'s for which this is true is typically small. 
%Hence \eqref{eqn:indicator} can be rewritten as a limitation of the sum over possible state transitions:
%\begin{equation}
%    \hat{\alpha}_t(z_t) = 
%    \begin{cases} 
%            p(y_t|z_t)\sum_{z_{t-1}}\hat{\alpha}_{t-1}(z_{t-1}) & \textrm{if $u_t < p(z_t | z_{t-1})$} \\
%            0 & \textrm{otherwise}
%        \end{cases}    
%    \label{eqn:scaled forward}
%\end{equation}
%which has restricted the calculation to a finite set. 
%Care must be taken to find those transitions $p(z_t | z_{t-1})$ whose likelihood is greater than the auxiliary variable $u_t$, though this is a straightforward task for unimodal distributions.

%The auxiliary variable encodes all of the information we have about the probability of making a transition: most of the time the only valid transition will be to decrement $d_t$ and to stay in the same state. At the end of a segment the potential transitions will be filtered using the auxiliary variable, discarding unlikely transitions. The actual transition will be chosen from the remaining set, depending on the observation likelihood associated with each transition. The slightly un-intuitive aspect of this is that once this filtering has taken place, the transition probabilities are no longer used in the determination of the transition, unlike in standard forward-backward algorithms.

The backwards sampling step recursively samples a state sequence from the distribution $p(z_{t-1} | z_{t}, \mathcal{Y}, \mathcal{U})$
%\begin{equation}
%    p(z_t | z_{t+1}, \mathcal{Y}, \mathcal{U}).
%\end{equation}
which can expressed in terms of the forward variable:
\begin{eqnarray}
    p(z_{t-1} | z_{t}, \mathcal{Y}, \mathcal{U}) &\propto& p(z_{t},z_{t-1}, \mathcal{Y}, \mathcal{U})  \label{eqn:backward} \\
    & \propto &  
        p(u_{t} | z_t, z_{t-1})p(z_{t}|z_{t-1}) \hat{\alpha}_{t-1}(z_{t-1})
        %p(z_t, \mathcal{Y}_1^t, \mathcal{U}_1^t)
        %p(\mathcal{Y}_{t+1}^T, \mathcal{U}_{t+2}^T | z_{t+1}) 
        \nonumber\\
    & \propto & 
       \mathbb{I}(0 < u_{t} < p(z_{t} | z_{t-1}))
        \hat{\alpha}_{t-1}(z_{t-1}).\nonumber
\end{eqnarray}

%Note that the indicator function used in the backwards sampler further restricts possible transitions. This restriction is conditional on the sampled $z_{t+1}$, limiting the transitions selected by in the forwards to those that could possibly result in $z_{t+1}$.

The full EDHMM beam sampler is given in Algorithm \ref{alg:beam}, which makes use of the forward recursion in \eqref{eqn:scaled forward}, the slice sampler in \eqref{eqn:slice}, and the backwards sampler in \eqref{eqn:backward}.




\subsection{Related Work}

The need to accommodate explicit state duration distributions in HMMs has long been recognised. Rabiner \cite{Rabiner89} details the basic approach which expands the state space to include dwell time before applying a slightly modified Baum-Welch algorithm. This approach specifies a maximum state duration, limiting practical application 
to cases with short sequences and dwell times.
This approach, generalised under the name ``segmental hidden Markov models'', includes more general transitions than those Rabiner considered, allowing the next state and duration to be conditioned on the previous state and duration \cite{Gales93}. Efficient approximate inference procedures were developed in the context of speech recognition \cite{Ostendorf96}, speech synthesis \cite{Zen07}, and evolved into symmetric approaches suitable for practical implementation \cite{Yu2006}. Recently, a ``sticky'' variant of the hierarchical Dirichlet process HMM (HDP-HMM) has been developed \cite{Fox2008}.  The HDP-HMM has countable state-cardinality \cite{Teh06} allowing estimation of the number of states in the HMM; the sticky aspect addresses long dwell times by introducing a parameter in the prior that favours self-transition. 


\section{Experiments}

\label{sec:experiments}


\begin{figure}
    \subfloat[][]{
        \includegraphics[width=0.5\textwidth]{../pic/experiment_2_X.pdf}
        \label{fig:exp_1_state}
    } \\
    \subfloat[][]{
        \includegraphics[width=0.5\textwidth]{../pic/experiment_2_Y.pdf}
        \label{fig:exp_1_data}
    }
    \caption{Example a) state and b) observation sequence generated by the explicit duration HMM. Here $K$ = 3; $p(y_t|x_t=j) = \mathrm{N}(\mu_j, 1)$ with $\mu_1 = -3$, $\mu_2 = 0$, and $\mu_3 = 3$; and $p(d_t|x_t=j) = \mathrm{Poisson}(\lambda_j)$ with $\lambda_1 = 5$, $\lambda_2 = 15$, and $\lambda_3 = 20$.}
    \label{fig:experiment1_data}
\end{figure}

\begin{figure}
    \subfloat[][]{
        \includegraphics[width=0.25\textwidth]{../pic/posterior_means.pdf}
        \label{fig:posterior_means}
    } 
    \subfloat[][]{
        \includegraphics[width=0.25\textwidth]{../pic/posterior_rates.pdf}
        \label{fig:posterior_rates}
    }
    \caption{Samples from the posterior distributions of a) the observation distribution means and b) the duration distribution rate parameters.}
    \label{fig:experiment1_results}
\end{figure}
\subsection{Synthetic Data}

The first experiment uses the 500 data points shown in Figure \ref{fig:experiment1_data} generated from an EDHMM with three states. To generate the data the duration distributions were chosen to be Poisson with rates $\lambda_1 = 5$, $\lambda_2 = 15$, $\lambda_3 = 20$; each observation distribution was Gaussian with means of $\mu_1 = -3$, $\mu_2 = 0$, and $\mu_3 = 3$, each with a variance of 1. The transition distributions $A$ were set to
\begin{equation*}
\begin{bmatrix}
    0 & 0.3 & 0.7 \\ 0.6 & 0 & 0.4 \\ 0.3 & 0.7 & 0
\end{bmatrix}.  
\end{equation*}



For inference, broad, uninformative priors were chosen for the parameters of the duration and observation distributions. The parameters of the observation distribution for all states were given a normal-inverse-Wishart (N-IW) prior with parameters $\nu_0 = 2$, $\Lambda_0 = 1$, $\kappa=0.1$ and $\mu_0 = 0$. The rate parameters for all states were given $\mathrm{Gamma}(1, 10^{5})$ priors. 

One thousand samples were collected from the EDHMM beam sampler after a burn-in of 500 samples. The learned posterior distribution of the state duration parameters and means of the observation distributions are shown in Figure \ref{fig:experiment1_results}.  The EDHMM achieves high accuracy in the estimated posterior distribution of the observation means, despite the overlap in observation distributions. The rate parameter distributions are reasonably estimated given the small number of observed segments.

A second experiment was performed using similar synthetic data to demonstrate the ability of the EDHMM to distinguish between states having differing duration distributions but the same observation distribution. Here the same model was used as above except data 
%was 
were
generated using $\mu_1 = 0$, $\mu_2 = 0$, and $\mu_3 = 3$. The same priors were used as above, and again the sampler was allowed to run for 1000 samples with a burn-in of 500 samples.

Figure~\ref{fig:experiment2_results} shows that the sampler clearly separates the high state associated with $\mu_3$ from the other two states and clearly 
%identifies the fact that there are 
reveals the presence of
two  low states with differing duration distributions.   That it is possible for the EDHMM to recover two states with differing duration distributions requires a third state.  Figure~\ref{fig:exp_2_state} shows posterior samples that indicate that the model is mixing over ambiguities about states $0$ and $1$ as it should. %As shown in Figure \ref{fig:experiment2_results}, when the state transitions to and from the long-high state to the short-low state (e.g. $t\approx 100$ and $t\approx 180$) the transitions are captured well. When the system transitions between the two low states the segmentation is less accurate, due to the inherent lack of identifiability during these periods. Instead, the space of valid state sequences is explored, each sequence obeying the duration distribution.

%Despite the loss of identifiability in the state sequence, the sampler is still able to capture reasonable estimates for the parameters of the observation and duration distributions.

\begin{figure}
    \subfloat[][]{
        \includegraphics[width=0.5\textwidth]{../pic/experiment_3_Y.pdf}
        \label{fig:exp_2_data}
    } \\
    \subfloat[][]{
        \includegraphics[width=0.5\textwidth]{../pic/experiment_3_X.pdf}
        \label{fig:exp_2_state}
    } \\
    \subfloat[][]{
        \includegraphics[width=0.25\textwidth]{../pic/posterior_means_exp3.pdf}
        \label{fig:posterior_means_3}
    } 
    \subfloat[][]{
        \includegraphics[width=0.25\textwidth]{../pic/posterior_rates_exp3.pdf}
        \label{fig:posterior_rates_3}
    }
    \caption{Results of the beam sampler applied to a system that generates states with identical observation distribution but differing durations. The observations are shown in a), and the true states are shown in b) overlaid with 20 randomly selected state traces produced by the sampler after burn-in. Samples from the posterior observation distribution mean are shown in c), and samples from the posterior duration distribution rates are shown in d).}
    \label{fig:experiment2_results}
\end{figure}



Finally, Figure \ref{fig:allowed} shows that the number of allowed transitions decays with the number of iterations, significantly speeding up the runtime of the later samples after the burn-in period. 


\begin{figure}
    \includegraphics[width=0.5\textwidth]{../pic/number_transitions_visited.pdf}

\caption{Mean number of transitions visited over the first 1000 iterations of the beam sampler.}
\label{fig:allowed}
\end{figure}



\section{Discussion}

\label{sec:dicussion}

We presented a beam sampler for the explicit state duration HMM. This sampler draws state sequences from the true posterior distribution without any need to make truncation approximations.
% introduction of an auxiliary variable sampling technique stochastically limits the possible transitions to a finite set, whose size is governed by the effective support of the duration distribution. Hence the computational complexity of the algorithm is governed by the support of the duration distribution instead of the magnitude of the duration itself.  
It remains future work to combine the explicit state duration HMM and the iHMM.
Python code associated with the EDHMM is available online.\footnote{http://github.com/mikedewar/EDHMM}

%Though it has not been the focus of this paper, the sampling technique could be readily extended to an infinite variant of the EDHMM, combining an infinite state representation with the infinite duration representation presented above. For more complex behaviours, the beam sampler could be applied to the SLDS to capture long behaviours without suffering a computational penalty.

\bibliographystyle{plain}

\bibliography{../inc/library}

\end{document}
