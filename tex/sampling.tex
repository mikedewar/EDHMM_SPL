Our aim is to estimate the conditional posterior distribution of the latent states ($\mathcal{X}$ and $\mathcal{D}$) and parameters ($\theta, \lambda$ and $A$) given observations $\mathcal{Y}$ by samples drawn via Markov chain Monte Carlo. Sampling $\theta$ and $A$ given $\mathcal{X}$ proceeds per usual textbook approaches \cite{Bishop06}.  Sampling $\lambda$ given $\mathcal{D}$ is straightforward in most situations.  Indirect Gibbs sampling  of $\mathcal{X}$ is possible using auxiliary state-change indicator variables, but for reasons similar to those in \cite{Goldwater2009}, such a sampler will not mix well.  The main contribution of this paper is to show how to generate posterior samples of  $\mathcal{X}$ and $\mathcal{D}$.

\subsection{Forward Filtering, Backward Sampling}

We can, in theory, use the forward messages from the forward backward algorithm \cite{Rabiner89} to sample the conditional posterior distribution of $\mathcal{X}$ and $\mathcal{D}.$   To do this we treat each state-duration tuple as a single random variable 
%(i.e.~$z_t = \{x_t,d_t\}$).  
(introducing the notation $z_t = \{x_t,d_t\}$).  
Doing so recovers the standard hidden Markov model structure and hence standard forward messages can be used directly.  A forward filtering, backward sampler for $\mathcal{Z} = \{z_1, \ldots, z_T\}$ conditioned on 
%everything else 
all other random variables
requires the classical forward messages:
    \begin{equation}
        \alpha_t(z_t) = 
        \sum_{z_{t-1}}
        p(z_t | z_{t-1}) 
        p(y_t|z_t) 
        \alpha_t(z_{t-1}) 
        \label{eqn:forward recursion}
    \end{equation}
   % \begin{equation}
   % \beta(z_t) = \sum_{z_{t+1}} p(z_{t+1} | z_t) p(y_{t+1}|z_{t+1}) \beta_{t+1}(z_{t+1})
   % \label{eqn:backward recursion}
    % \end{equation}
     where the transition probability can be factorised according to our modelling assumptions:
     \begin{equation}
        p(z_{t} | z_{t-1}) = p(x_t | x_{t-1}, d_{t-1}) p(d_t | d_{t-1}, x_t).
     \end{equation}

Unfortunately the sum in \eqref{eqn:forward recursion} has at worst an infinite number of terms in the case of  duration distributions with countably infinite support and at best a very large number of terms in the case of long sequences. The standard approach to EDHMM inference involves truncating considered durations to only those that lie between $d_\mathrm{min}$ and $d_\mathrm{max}$ or computation involving all possible durations up to the observed length of the sequence ($d_\mathrm{min}=0, d_\mathrm{max}=T$). This leads to per-sample, forward-backward computational complexity of $O(T(K(d_\mathrm{max}-d_\mathrm{min}))^2)$.
Truncation yields inference that will simply fail if an actual duration lies outside hard-coded allowable duration intervals.   Considering all possible durations up to length $T$ is often computationally impossible.
%This yields $O(T(K(d_\mathrm{max}-d_\mathrm{min}))^2)$ complexity.
The beam-sampler we propose behaves like a dynamic version of the truncation approach, automatically defining and scaling per-state duration truncation intervals.   Better though, the way it does this results in an asymptotically exact sample with no risk of incorrect inference resulting from incorrectly pre-specified duration truncations.   We do not characterize the computational complexity of the proposed beam sampler in this work but note that it is upper bounded by $O(T(KT)^2)$ (i.e., the beam sampler admits durations of length equal to the entire sequence) but in practice is found to be as  or more efficient than the risky hard-truncation approach.

%would be achieved with a uniform duration distribution with support over $[1, d_\mathrm{max}]$. 

\subsection{EDHMM Beam Sampling}

A recent contribution to inference 
in 
%is 
the infinite Hidden Markov Model (iHMM) \cite{Beal2002} suggests a way around truncation \cite{vanGael2008}.  The iHMM is an HMM with a countable number of states.  Computing the forward message for a forward filtering, backward sampler for the latent states in an iHMM also requires a sum over a countable number of elements.  
% which can be dealt with using the beam sampler \cite{vanGael2008}. Here, instead of a %potentially infinite duration-space cardinality, the iHMM faces a potentially infinite state cardinality. 
%The beam sampler combines dynamic programming with slice sampling \cite{Neal2003} in order to sample from the hierarchical Dirichlet process inherent to the iHMM.
The ``beam sampling'' approach  \cite{vanGael2008}, which we can apply largely without modification, is to truncate this sum by introducing a ``slice'' \cite{Neal2003} auxiliary variable $\mathcal{U} = \{u_1, u_2, \ldots,u_T\}$ at each time step.  The auxiliary variables are chosen in such a way as to automatically limit each sum in the forward pass to a finite number of terms while still allowing all possible durations.% in the asymptotic sampling limit.  

%which is used in a Gibbs sampling scheme to allow sampling from the true posterior state transition distribution.
%In the context of the EDHMM, the beam sampler allows us to sample directly from the invariant posterior $p(\mathcal{X},\mathcal{D},\mathcal{U} | \mathcal{Y})$ using standard Gibbs sampling theory. Using a block-Gibbs sampling scheme, $\mathcal{U}$ is sampled so that we are able to sample from $p(\mathcal{X},\mathcal{D} | \mathcal{U}, \mathcal{Y})$ and $p(\mathcal{U} | \mathcal{X},\mathcal{D})$. Upon convergence of the Markov chain, by simply marginalising $\mathcal{U}$ the remaining samples of state and duration are drawn directly from the posterior, without any need to choose a maximum duration. 

The particular choice of auxiliary variable $u_t$ is important.  We follow  \cite{vanGael2008} in choosing $u_t$ to be conditionally distributed given the current and previous state and duration in the following way (see the graphical model in Figure \ref{fig:aux graphical model}):
\begin{equation}
    \label{eqn:slice}
    p(u_t | z_t, z_{t-1}) = 
    \frac
    {\mathbb{I}(0 < u_t < p(z_t | z_{t-1}))} 
    {p(z_t | z_{t-1})}.
\end{equation}
where $\mathbb{I}(\cdot)$ returns one if its operand is true and zero otherwise. Given $\mathcal{U}$ it is possible to sample the state $\mathcal{X}$ and duration $\mathcal{D}$ conditional posterior. 
%\begin{equation}
%    \hat{\alpha}_t(z_t) = p(z_t, d_t , \mathcal{Y}_1^t, \mathcal{U}_1^t)
%\end{equation}
Using notation $\mathcal{Y}_{t_1}^{t_2} = \{y_{t_1}, y_{t_1+1}, \ldots,y_{t_2}\}$  to indicate sub-ranges of a sequence, the new forward messages we compute are:
\begin{eqnarray}
   \hat{\alpha}_t(z_t) &=& 
   p(z_t, \mathcal{Y}_1^t , \mathcal{U}_1^{t})   \label{eqn:scaled forward} \\
   &=& 
   \sum_{z_{t-1}}
   p(z_t, z_{t-1} , \mathcal{Y}_1^t , \mathcal{U}_1^{t}) \nonumber \\
   &\propto& 
   \sum_{z_{t-1}}
   p(u_{t} | z_t, z_{t-1})
   p(z_t, z_{t-1} , \mathcal{Y}_1^t, \mathcal{U}_1^{t-1}) \nonumber \\
   &=& 
   \sum_{z_{t-1}}
   \mathbb{I}(0 < u_{t} < p(z_t | z_{t-1}))
   p(y_t|z_t) \hat{\alpha}_{t-1}(z_{t-1}) \nonumber.
\end{eqnarray}
The indicator function $\mathbb{I}$ results in non-zero probabilities in the forward message for only those states $z_t$ whose likelihood given $z_{t-1}$ is greater than $u_t$. The beam sampler derives its computational advantage from the fact that the set of $z_t$'s for which this is true is typically small. 
%Hence \eqref{eqn:indicator} can be rewritten as a limitation of the sum over possible state transitions:
%\begin{equation}
%    \hat{\alpha}_t(z_t) = 
%    \begin{cases} 
%            p(y_t|z_t)\sum_{z_{t-1}}\hat{\alpha}_{t-1}(z_{t-1}) & \textrm{if $u_t < p(z_t | z_{t-1})$} \\
%            0 & \textrm{otherwise}
%        \end{cases}    
%    \label{eqn:scaled forward}
%\end{equation}
%which has restricted the calculation to a finite set. 
%Care must be taken to find those transitions $p(z_t | z_{t-1})$ whose likelihood is greater than the auxiliary variable $u_t$, though this is a straightforward task for unimodal distributions.

%The auxiliary variable encodes all of the information we have about the probability of making a transition: most of the time the only valid transition will be to decrement $d_t$ and to stay in the same state. At the end of a segment the potential transitions will be filtered using the auxiliary variable, discarding unlikely transitions. The actual transition will be chosen from the remaining set, depending on the observation likelihood associated with each transition. The slightly un-intuitive aspect of this is that once this filtering has taken place, the transition probabilities are no longer used in the determination of the transition, unlike in standard forward-backward algorithms.

The backwards sampling step recursively samples a state sequence from the distribution $p(z_{t-1} | z_{t}, \mathcal{Y}, \mathcal{U})$
%\begin{equation}
%    p(z_t | z_{t+1}, \mathcal{Y}, \mathcal{U}).
%\end{equation}
which can expressed in terms of the forward variable:
\begin{eqnarray}
    p(z_{t-1} | z_{t}, \mathcal{Y}, \mathcal{U}) &\propto& p(z_{t},z_{t-1}, \mathcal{Y}, \mathcal{U})  \label{eqn:backward} \\
    & \propto &  
        p(u_{t} | z_t, z_{t-1})p(z_{t}|z_{t-1}) \hat{\alpha}_{t-1}(z_{t-1})
        %p(z_t, \mathcal{Y}_1^t, \mathcal{U}_1^t)
        %p(\mathcal{Y}_{t+1}^T, \mathcal{U}_{t+2}^T | z_{t+1}) 
        \nonumber\\
    & \propto & 
       \mathbb{I}(0 < u_{t} < p(z_{t} | z_{t-1}))
        \hat{\alpha}_{t-1}(z_{t-1}).\nonumber
\end{eqnarray}

%Note that the indicator function used in the backwards sampler further restricts possible transitions. This restriction is conditional on the sampled $z_{t+1}$, limiting the transitions selected by in the forwards to those that could possibly result in $z_{t+1}$.

The full EDHMM beam sampler is given in Algorithm \ref{alg:beam}, which makes use of the forward recursion in \eqref{eqn:scaled forward}, the slice sampler in \eqref{eqn:slice}, and the backwards sampler in \eqref{eqn:backward}.

